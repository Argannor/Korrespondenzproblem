\documentclass[]{scrartcl}


\usepackage[ngerman]{babel}
\usepackage{default}
\usepackage[utf8]{inputenc}
\usepackage{graphicx}

\usepackage{amsmath}
\usepackage{amssymb}
\usepackage{amsthm}

\newtheorem{definition}{Definition}[section]
\newtheorem{satz}[definition]{Satz}
\newtheorem{bsp}[definition]{Beispiel}

%opening
\title{Postsches Korrespondenzproblem}
\author{Soeren Berken-Mersmann}

\begin{document}

\maketitle

\begin{abstract}

\end{abstract}

\section{Einleitung}

\section{Postsches Korrespondenzproblem}

	\subsection{Definition}
	
		Das Postsches Korrespondenzproblem (PKP) ist benannt nach Emil Leon Post. Es handelt sich um ein konstruiertes Problem aus der theoretischen Informatik und ist folgendermaßen definiert:
		
		\begin{definition}
			Gegeben sei eine endliche Menge an Wortpaaren $K = ((x_1, y_1), ..., (x_k, y_k))$, über dem Alphabet $\Sigma$ mit $x_i, y_i \in \Sigma$. Gibt es eine Folge von Indizes $i_1, i_2, ..., i_n \in {1, 2, ..., k}, n \geq 1$, so dass $x_{i_1},x_{i_2}, ... x_{i_n} = y_{i_1}, y_{i_2}, ..., y_{i_n}$?
		\end{definition}
		
		Das Problem lässt sich besser Visualisieren als Menge von Dominosteinen: Ein Dominostein besteht jeweils aus einem oberen ($x_i$) und unteren ($y_i$) Symbol aus dem Alphabet $\Sigma$. Die Fragestellung lautet nun, ob es eine Reihenfolge gibt in der die Dominosteine so aneinandergereiht werden können, dass oben und unten die selbe Symbolfolge entsteht. Dabei darf jeder Dominostein mehrfach verwendet werden. Diese Arbeit wird daher eine Schreibweise verwenden, die diese Visualisierung widerspiegelt.
	
	\subsection{Beispielinstanzen}
		
		Dieser Abschnitt wird an Hand von drei Beispielen zeigen, dass die Lösung des PKPs nicht trivial ist.

		\begin{bsp}
			\label{bsp-pkp1}
			Die Menge an Wortpaaren $K$ sei gegeben mit: \[K = \left\lbrace \begin{bmatrix}
					1 \\ 111
				\end{bmatrix}
				\begin{bmatrix}
					10111 \\ 10
				\end{bmatrix}
				\begin{bmatrix}
					10 \\ 0
				\end{bmatrix}\right\rbrace \]
				Die Lösung des PKPs lässt sich mit der Indexfolge $I_1 = (2, 1, 1, 3)$ angeben, so ergibt sich die Lösung:
				\[K_{I_1} = \begin{bmatrix}
								10111 \\ 10
							\end{bmatrix} \begin{bmatrix}
								1 \\ 111
							\end{bmatrix} \begin{bmatrix}
								1 \\ 111
							\end{bmatrix} \begin{bmatrix}
								10 \\ 0
							\end{bmatrix}\]
		\end{bsp}
		
		Das Beispiel \ref{bsp-pkp1} zeigt, dass es Instanzen des PKPs gibt für die sich eine Lösung finden lässt. Darüber hinaus handelt es sich um ein einfaches Beispiel, bei dem man mit ein wenig Knobelei selbst auf die Lösung kommen kann. Mit Hilfe von zwei weiteren Beispielen wird gezeigt, dass es sowohl weitaus komplexere Probleminstanzen gibt, die sich nicht mehr mit einem scharfen Blick lösen lassen (s. Beispiel \ref{bsp-pkp2}), als auch Fälle in denen es keine Lösung für das PKP gibt (Beispiel \ref{bsp-pkp3}).
		
		\begin{bsp}
			\label{bsp-pkp2}
			Die Instanz des PKPs sei gegeben mit der Menge an Wortpaaren $K$:
			\[K = \left\lbrace \begin{bmatrix}
						001 \\ 0
					\end{bmatrix}
					\begin{bmatrix}
						01 \\ 011
					\end{bmatrix}
					\begin{bmatrix}
						01 \\ 101
					\end{bmatrix}
					\begin{bmatrix}
						10 \\ 001
					\end{bmatrix}\right\rbrace \]
			Und ihre kürzeste Lösung ist $I_1$ mit insgesamt 66 Elementen:
			\begin{align*}
				I_1 = (2, 4, 3, 4, 4, 2, 1, 2, 4, 3, 4, 3, 4, 4, 3, 4, 4, 2, 1, 4, 4, 2, 1, 3, 4, 1, 1, 3, 4, 4, 4, 2, 1,\\ 2, 1, 1, 1, 3, 4, 3, 4, 1, 2, 1, 4, 4, 2, 1, 4, 1, 1, 3, 4, 1, 1, 3,
					1, 1, 3, 1, 2, 1, 4, 1, 1, 3 )
			\end{align*}
		\end{bsp}
		\begin{bsp}
			\label{bsp-pkp3}
			Die Instanz des PKPs sei gegeben mit der Menge an Wortpaaren $K$:
			\[K = \left\lbrace \begin{bmatrix}
						10 \\ 101
					\end{bmatrix}
					\begin{bmatrix}
						011 \\ 11
					\end{bmatrix}
					\begin{bmatrix}
						101 \\ 011
					\end{bmatrix}\right\rbrace \]
			Das einzige Wortpaar mit dem angefangen werden kann ist das erste. Darauf muss zwangsläufig das dritte Paar folgen.  Da nun die untere Wortfolge der oberen um ein Symbol vorauseilt, passt erneut nur das dritte Paar. Folglich kann die $y$-Folge nie aufschließen, und diese Instanz des PKPs hat keine Lösung.
			\[
			\begin{bmatrix}
						10 \\ 101
					\end{bmatrix} 
					\begin{bmatrix}
						101 \\ 011
					\end{bmatrix}
					\begin{bmatrix}
						101 \\ 011
					\end{bmatrix} ...
			\]
		\end{bsp}

\section{Beweis der Nichtberechenbarkeit}
	In diesem Kapitel soll die Nichtentscheidbarkeit des PKPs bewiesen werden. Dazu wird das Halteproblem für Turingmaschinen zu Hilfe genommen.

	\subsection{Simulation einer Turingmaschine}
		Dazu soll zuerst gezeigt werden, wie mit Hilfe des PKPs eine Turingmaschine simuliert werden kann. Dabei wird eine Turingmaschine mit einem beidseitig unendlichem Band und dem Bandalphabet $\varGamma = {0,1}$ verwendet, ohne die Allgemeinheit zu beschränken. Um die Turingmaschine mit Hilfe des PKPs simulieren zu können, muss zuerst ihr Rechenweg formalisiert werden. 
		
		
		\paragraph{Der Rechenweg einer Turingmaschine}
			Der Rechenweg beschreibt dabei die von der Turingmaschine durchlaufenen Zustände\footnote{Mit Zuständen wird der gesamte Zustand der Turingmaschine gemeint, und nicht der Zustand des Turingprogramms. Der Verständlichkeit halber wird dieser Programmzustand im weiteren als interner Zustand bezeichnet.} vom Beginn der Berechnung bis zu einem (möglichen) Endzustand. Der Rechenweg ist folglich eine Folge von Zuständen.
			
			Der Zustand einer Turingmaschine lässt sich mit den folgenden Punkten vollständig beschreiben (Vgl. Abbildung \ref{img-turingsnapshot}):
			\begin{figure}
				\centering
				\includegraphics[width=0.85\linewidth]{../abbildungen/turing-snapshot}
				\caption{Illustration des Zustands einer Turingmaschine mit beidseitig unendlichem Band.}
				\label{img-turingsnapshot}
			\end{figure}
			
			\begin{itemize}
				\item Linkskontext: $u$
				\item Interner Zustand: $q$
				\item Gelesenes Symbol: $a$
				\item Rechtskontext: $w$
				\item Position des Lese- und Schreibkopfes
			\end{itemize}
			
			\begin{satz}
			Somit lässt sich der Zustand $Q_t$ einer Turingmaschine zum Zeitpunkt $t$ durch die Folge $Q_t = u_tq_ta_tw_t$ darstellen. Dabei ist die Position des Lese- und Schreibkopfes durch die Position des internen Zustands innerhalb der Folge $Q_t$ kodiert. Folglich lassen sich das gelesene Symbol $a_t$ und der Rechtskontext $w_t$ auch zusammenfassen.
			\end{satz}
			
			\begin{satz}
			Den Rechenweg einer Turingmaschine können wir als die Folge von Zuständen $Q_0, ..., Q_n$ vom Startzeitpunkt $t = 0$ bis zum Endzeitpunkt $t = n$ bei dem die Turingmaschine einen der Endzustände erreicht hat.
			\end{satz}
		
		\paragraph{Simulation mit Hilfe des PKPs}
			
			

	\subsection{Reduktion des Halteproblems}

\section{Beweise mit Hilfe des PKPs}

	\subsection{Eindeutigkeitsfrage}

	\subsection{Äquivalenzproblem}

\end{document}
